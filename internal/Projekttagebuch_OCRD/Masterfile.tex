\documentclass[12pt, a4paper, parskip=half, bibliography=openstyle, DIV=calc]{scrartcl}
\usepackage[left= 2cm,right = 2cm, bottom = 4.5cm, top = 2.5cm]{geometry}

\usepackage[
	pdftitle={Titel der Abschlussarbeit},
	pdfsubject={},
	pdfauthor={Markus Frank},
	pdfkeywords={},	
	colorlinks=true, %Link-Package mit gleichzeitiger Einfärbung der Links
	breaklinks=true,
	citecolor=red,
	linkcolor=blue,
	menucolor=black,
	urlcolor=blue,
	breaklinks=true	
]{hyperref}



% Standard Packages
\usepackage[onehalfspacing]{setspace}
\usepackage[utf8]{inputenc}
\usepackage[ngerman]{babel} %Zeilentrennung etc.
\usepackage[T1]{fontenc} %Schrift-Kodierung
\usepackage{lmodern} %Schönere Schrift
\usepackage{graphicx, subfig}
\graphicspath{{img/}} %Grafikpfad
\usepackage{fancyhdr}
\usepackage{xcolor} %Verbessertes Color-Packet
\usepackage{transparent}
\usepackage[german=guillemets]{csquotes} %Zitatpacket. Zitate mit \enquote{INHALT}
\usepackage{booktabs}
\usepackage{amssymb}
\usepackage{float}
\usepackage{multirow}
\usepackage{colortbl}
\usepackage{minibox}
\usepackage{varwidth}
% zusätzliche Schriftzeichen der American Mathematical Society
\usepackage{amsmath} %Mathematisches Modul
\usepackage{amsthm} % Modul für Theoreme
\usepackage{amsfonts} %Mathematisches Modul
\usepackage{tikz}

% Spezielle Packages für die Promotion
\usepackage[printonlyused]{acronym} %Erzeugt Abkürzungsverzeichnis
\usepackage{covington} %Für linguistische Beispiele
\usepackage{paralist} %Besondere Listentypen
\usepackage{listings} %Für die Eingabe von Programmcode
%\usepackage {picins} %Ermöglicht Positionierung von Bildern in Gleitumgebungen.
\usepackage{longtable} %Tabellen über mehrere Seiten.
\usepackage{hyperref}


\definecolor{codegreen}{rgb}{0,0.4,0}
\definecolor{codegray}{rgb}{0.5,0.5,0.5}
\definecolor{codepurple}{rgb}{0.58,0,0.82}
\definecolor{backcolour}{rgb}{0.95,0.95,0.92}
 
% Definition der Farben für die Programmiersprachen
\lstdefinestyle{mystyle}{
    backgroundcolor=\color{backcolour},   
    commentstyle=\color{codegreen},
    keywordstyle=\color{magenta},
    numberstyle=\tiny\color{codegray},
    stringstyle=\color{codepurple},
    basicstyle=\footnotesize,
    breakatwhitespace=false,         
    breaklines=true,                 
    captionpos=b,                    
    keepspaces=true,                 
    numbers=left,                    
    numbersep=5pt,                  
    showspaces=false,                
    showstringspaces=false,
    showtabs=false,                  
    tabsize=2
}
 
\lstset{style=mystyle}

\lstset{literate=%
    {Ö}{{\"O}}1
    {Ä}{{\"A}}1
    {Ü}{{\"U}}1
    {ß}{{\ss}}1
    {ü}{{\"u}}1
    {ä}{{\"a}}1
    {ö}{{\"o}}1
    {~}{{\textasciitilde}}1
}


\lstdefinestyle{sql}
{
    language=SQL,
    basicstyle=\small,
    identifierstyle=\ttfamily,
    keywordstyle=\bfseries\ttfamily\color[rgb]{0,0,1},
    stringstyle=\ttfamily\color[rgb]{0.627,0.126,0.941},
    commentstyle=\color{codegreen},
    numberstyle=\tiny\color{codegray},
    morekeywords={REFERENCES, LOAD, DATA, INFILE, FIELDS, TERMINATED, LINES, ENCLOSED, IGNORE, REGEXP, GROUP_CONCAT, LENGTH, CONCAT, CONCAT_WS, SEPARATOR, IF, STRCMP, WHEN, THEN, CASE, INSERT, INTO, VALUES, USE, DESCRIBE, DROP, TABLE, DATABASE, TRUNCATE, SHOW, TABLES, AUTO_INCREMENT}
}

\lstnewenvironment{sql}
{\lstset{style=Sql}}
{}



%Jedes Kapitel beginnt mit neuer Nummerierung
\makeatletter
\@addtoreset{chapter}{part}
\makeatother  

%Gliederungstiefe der Kapitel auf 5 Setzen
\setcounter{secnumdepth}{3}
\setcounter{tocdepth}{4}

\makeatletter
\@addtoreset{section}{part}
\makeatother

% -------------- Karomuster --------------------

\newcommand{\karos}[2]{
  \begin{tikzpicture}
    \draw[step=0.5cm,color=lightgray] (0,0) grid (#1 cm ,#2 cm);
  \end{tikzpicture}
}

% ============= Kopf- und Fußzeile =============
%Kopf und Fußzeile
\usepackage{fancyhdr}
\pagestyle{fancy}

%Kopfzeile Head
%\chead{\includegraphics[width=1\textwidth]{Header.png}}
\lhead{}
\rhead{}

\lfoot{\leftmark}
\cfoot{}
\rfoot{Seite \thepage}

\headheight 65pt
\renewcommand{\headrulewidth}{0pt}
\renewcommand{\footrulewidth}{1pt}




\begin{document}
\begin{NoHyper}

\begin{LARGE}
\textbf{Planung:} OCRD Modulprojekt 3:
\glqq Automatische Nachkorrektur historischer OCR-erfasster Drucke 
mit integrierter optionaler interaktiver Nachkorrektur\grqq{} 
\end{LARGE}

\tableofcontents
\end{NoHyper}

\section{Stand der Implementierungen}

\section{Meetings}
\subsection {Kick-Off-Treffen 05.3 - 06.3 (Wolfenbüttel)}

\begin{itemize}
  \item \href{https://wiki.de.dariah.eu/pages/viewpage.action?pageId=64949522}{Offizielles Protokoll}
  \item Schwerpunkt: Gemeinsame Datenformate (METS + PAGE) für Eingabe und Ausgabe.
\end{itemize}

\subsection {Videokonferenz  20.4}

\begin{itemize}
\item \href{https://wiki.de.dariah.eu/display/OCR/2018-04-20+Besprechungsnotizen}{Offizielles Protokoll
}
\item Stand des Modulprojekts erläutert.
\item Vorstellung der OCRD-CLI Schnittstelle.
\end{itemize}


\subsection {1. Entwicklertreffen 26.6 - 27.6 (Berlin)}
\begin{itemize}
  \item \href{https://wiki.de.dariah.eu/display/OCR/1.+Entwickler-Workshop}{Offizielles Protokoll}
  \item Schwerpunkt: Integration des Modulprojekts in OCRD Workflow / Workspace-Umgebung.
  \item Diskussion mit Leipzig:  
  \begin{itemize}
    \item Wie können Alignierungsergebnisse gemeinsam genutzt werden ?
      \begin{itemize}
    \item PAGE.xml als Austauschformat für Alignierungen multipler OCR mit GT.
    \item \texttt{Textequiv} zur Speicherung alternativer Alignierungen?
    \end{itemize}
	\item Gemeinsame Nutzung des neuronalen Sprachemodell
    \item Verständigung auf baldiges Treffen in München zur weiteren gemeinsamen Abstimmung.
  \end{itemize}
  \item Gespräch mit Kay:
  \begin{itemize}
    \item Frage nach der Erstellung von (multipler) OCR zu Testzwecken.
    \item Zeilenalignierung wird nicht benötigt, man kann davon ausgehen, dass Zeilen korrekt
    zueinander aligniert sind.
  \end{itemize}
  
\end{itemize}

\subsection {Treffen mit Robert und Lena 07.8 - 08.8 (München)}
\begin{itemize}
\item \href{https://git.informatik.uni-leipzig.de/ls36hiqo/ocr-d/wikis/planung#7-8-august-2018-am-cis-in-m%C3%BCnchen-mit-florian-und-tobias}{Protokoll}

\item Absprachen zum Format des PAGE.XMLs für Alignierung.
\item Gemeinsame Nutzung PAGE-Adapters.
\item Gemeinsame Nutzung der RNN Sprachmodelle.
\end{itemize}
\subsection {Videokonferenz  18.9}
\begin{itemize}
  \item \href{https://wiki.de.dariah.eu/display/OCR/2018-09-18+Besprechungsnotizen}{Offizielles Protokoll}
  \item Vorstellung Stand + Rückmeldung über positive Erfahrungen mit OCRD-Core Anbindung.
  \item Ansprechen der Problematik fehlender OCR-Beispieldaten für die Nachkorrektur.
  
\end{itemize}



\end{document}